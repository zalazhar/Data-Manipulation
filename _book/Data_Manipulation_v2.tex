\documentclass[12pt,]{book}
\usepackage{lmodern}
\usepackage{amssymb,amsmath}
\usepackage{ifxetex,ifluatex}
\usepackage{fixltx2e} % provides \textsubscript
\ifnum 0\ifxetex 1\fi\ifluatex 1\fi=0 % if pdftex
  \usepackage[T1]{fontenc}
  \usepackage[utf8]{inputenc}
\else % if luatex or xelatex
  \ifxetex
    \usepackage{mathspec}
  \else
    \usepackage{fontspec}
  \fi
  \defaultfontfeatures{Ligatures=TeX,Scale=MatchLowercase}
\fi
% use upquote if available, for straight quotes in verbatim environments
\IfFileExists{upquote.sty}{\usepackage{upquote}}{}
% use microtype if available
\IfFileExists{microtype.sty}{%
\usepackage{microtype}
\UseMicrotypeSet[protrusion]{basicmath} % disable protrusion for tt fonts
}{}
\usepackage[margin=1in]{geometry}
\usepackage{hyperref}
\hypersetup{unicode=true,
            pdftitle={THE ULTIMATE GUIDE TO DATA MANIPULATION WITH R AND PYTHON},
            pdfauthor={Zakaria Al Azhar, bigdatahabits.com},
            pdfborder={0 0 0},
            breaklinks=true}
\urlstyle{same}  % don't use monospace font for urls
\usepackage{natbib}
\bibliographystyle{apalike}
\usepackage{longtable,booktabs}
\usepackage{graphicx,grffile}
\makeatletter
\def\maxwidth{\ifdim\Gin@nat@width>\linewidth\linewidth\else\Gin@nat@width\fi}
\def\maxheight{\ifdim\Gin@nat@height>\textheight\textheight\else\Gin@nat@height\fi}
\makeatother
% Scale images if necessary, so that they will not overflow the page
% margins by default, and it is still possible to overwrite the defaults
% using explicit options in \includegraphics[width, height, ...]{}
\setkeys{Gin}{width=\maxwidth,height=\maxheight,keepaspectratio}
\IfFileExists{parskip.sty}{%
\usepackage{parskip}
}{% else
\setlength{\parindent}{0pt}
\setlength{\parskip}{6pt plus 2pt minus 1pt}
}
\setlength{\emergencystretch}{3em}  % prevent overfull lines
\providecommand{\tightlist}{%
  \setlength{\itemsep}{0pt}\setlength{\parskip}{0pt}}
\setcounter{secnumdepth}{5}
% Redefines (sub)paragraphs to behave more like sections
\ifx\paragraph\undefined\else
\let\oldparagraph\paragraph
\renewcommand{\paragraph}[1]{\oldparagraph{#1}\mbox{}}
\fi
\ifx\subparagraph\undefined\else
\let\oldsubparagraph\subparagraph
\renewcommand{\subparagraph}[1]{\oldsubparagraph{#1}\mbox{}}
\fi

%%% Use protect on footnotes to avoid problems with footnotes in titles
\let\rmarkdownfootnote\footnote%
\def\footnote{\protect\rmarkdownfootnote}

%%% Change title format to be more compact
\usepackage{titling}

% Create subtitle command for use in maketitle
\newcommand{\subtitle}[1]{
  \posttitle{
    \begin{center}\large#1\end{center}
    }
}

\setlength{\droptitle}{-2em}
  \title{THE ULTIMATE GUIDE TO DATA MANIPULATION WITH R AND PYTHON}
  \pretitle{\vspace{\droptitle}\centering\huge}
  \posttitle{\par}
  \author{Zakaria Al Azhar, bigdatahabits.com}
  \preauthor{\centering\large\emph}
  \postauthor{\par}
  \predate{\centering\large\emph}
  \postdate{\par}
  \date{2017-11-13}

\usepackage{booktabs}

\begin{document}
\maketitle

{
\setcounter{tocdepth}{1}
\tableofcontents
}
\chapter{Klm}\label{klm}

\chapter{Introduction}\label{intro}

Before you can really mine your data for insights you need to clean it
up. Even though it's always good practice to create a clean,
well-structured data set, sometimes it's not always possible. Data sets
can come in all shapes and sizes (some good, some not so good!),
especially when you're getting it from the web.

Data manipulation refers to a set of skills of changing data in an
effort to make it easier to read or be more organized, with the eventual
goal to get and present insight.

In this book we cover the most used data manipulation formulas, and show
how to implement the formulas with R and Python. The formulas are
classified in the following sections:

\begin{itemize}
\tightlist
\item
  \textbf{Column formulas}

  \begin{itemize}
  \tightlist
  \item
    \textbf{\emph{Binning}}: grouping data in intervals
  \item
    \textbf{\emph{Convert and Replace}}: casting datatypes, converting,
    replacing and column renaming.
  \item
    \textbf{\emph{Filter}}: Inclusion, Exclusion, Selecting, and
    Searching.
  \item
    \textbf{\emph{Split and Combine}}: Splitting, Aggregating,
    Combining, Merging, Joining, and Appending
  \item
    \textbf{\emph{Transform}}: Converting, Comparing, Resorting,
    Lagging, Missing Value, Normalizing/Denormalizing, One To Many, Many
    to One, String manipulation, Subsetting.
  \end{itemize}
\item
  \textbf{Row formulas}

  \begin{itemize}
  \tightlist
  \item
    \textbf{\emph{Filter}}: Row filter, and Data Splitting
  \item
    \textbf{\emph{Transform}}: Concatenate, Group by, Ungroup,
    Partitioning, Pivoting, Unpivioting, Sampling, and Sorting.
  \end{itemize}
\end{itemize}

\chapter{Prerequisites}\label{prerequisites}

\section{Basic knowledge}\label{basic-knowledge}

\section{Dataset}\label{dataset}

\section{Python and R packages}\label{python-and-r-packages}

\chapter{Data Exploration}\label{data-exploration}

\section{Data Structure}\label{data-structure}

\section{Data Summary}\label{data-summary}

\chapter{Column Formulas}\label{column-formulas}

\section{Data Binning}\label{data-binning}

\section{Convert \& Replace}\label{convert-replace}

\subsection{Category to Number}\label{category-to-number}

\chapter{Final Words and this is
strange}\label{final-words-and-this-is-strange}

\chapter{Placeholder}\label{placeholder}

\bibliography{packages.bib,book.bib}


\end{document}
