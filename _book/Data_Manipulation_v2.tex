\documentclass[12pt,]{book}
\usepackage{lmodern}
\usepackage{amssymb,amsmath}
\usepackage{ifxetex,ifluatex}
\usepackage{fixltx2e} % provides \textsubscript
\ifnum 0\ifxetex 1\fi\ifluatex 1\fi=0 % if pdftex
  \usepackage[T1]{fontenc}
  \usepackage[utf8]{inputenc}
\else % if luatex or xelatex
  \ifxetex
    \usepackage{mathspec}
  \else
    \usepackage{fontspec}
  \fi
  \defaultfontfeatures{Ligatures=TeX,Scale=MatchLowercase}
\fi
% use upquote if available, for straight quotes in verbatim environments
\IfFileExists{upquote.sty}{\usepackage{upquote}}{}
% use microtype if available
\IfFileExists{microtype.sty}{%
\usepackage{microtype}
\UseMicrotypeSet[protrusion]{basicmath} % disable protrusion for tt fonts
}{}
\usepackage[margin=1in]{geometry}
\usepackage{hyperref}
\hypersetup{unicode=true,
            pdftitle={THE ULTIMATE GUIDE TO DATA MANIPULATION WITH R AND PYTHON},
            pdfauthor={Zakaria Al Azhar, bigdatahabits.com},
            pdfborder={0 0 0},
            breaklinks=true}
\urlstyle{same}  % don't use monospace font for urls
\usepackage{natbib}
\bibliographystyle{apalike}
\usepackage{color}
\usepackage{fancyvrb}
\newcommand{\VerbBar}{|}
\newcommand{\VERB}{\Verb[commandchars=\\\{\}]}
\DefineVerbatimEnvironment{Highlighting}{Verbatim}{commandchars=\\\{\}}
% Add ',fontsize=\small' for more characters per line
\usepackage{framed}
\definecolor{shadecolor}{RGB}{248,248,248}
\newenvironment{Shaded}{\begin{snugshade}}{\end{snugshade}}
\newcommand{\KeywordTok}[1]{\textcolor[rgb]{0.13,0.29,0.53}{\textbf{#1}}}
\newcommand{\DataTypeTok}[1]{\textcolor[rgb]{0.13,0.29,0.53}{#1}}
\newcommand{\DecValTok}[1]{\textcolor[rgb]{0.00,0.00,0.81}{#1}}
\newcommand{\BaseNTok}[1]{\textcolor[rgb]{0.00,0.00,0.81}{#1}}
\newcommand{\FloatTok}[1]{\textcolor[rgb]{0.00,0.00,0.81}{#1}}
\newcommand{\ConstantTok}[1]{\textcolor[rgb]{0.00,0.00,0.00}{#1}}
\newcommand{\CharTok}[1]{\textcolor[rgb]{0.31,0.60,0.02}{#1}}
\newcommand{\SpecialCharTok}[1]{\textcolor[rgb]{0.00,0.00,0.00}{#1}}
\newcommand{\StringTok}[1]{\textcolor[rgb]{0.31,0.60,0.02}{#1}}
\newcommand{\VerbatimStringTok}[1]{\textcolor[rgb]{0.31,0.60,0.02}{#1}}
\newcommand{\SpecialStringTok}[1]{\textcolor[rgb]{0.31,0.60,0.02}{#1}}
\newcommand{\ImportTok}[1]{#1}
\newcommand{\CommentTok}[1]{\textcolor[rgb]{0.56,0.35,0.01}{\textit{#1}}}
\newcommand{\DocumentationTok}[1]{\textcolor[rgb]{0.56,0.35,0.01}{\textbf{\textit{#1}}}}
\newcommand{\AnnotationTok}[1]{\textcolor[rgb]{0.56,0.35,0.01}{\textbf{\textit{#1}}}}
\newcommand{\CommentVarTok}[1]{\textcolor[rgb]{0.56,0.35,0.01}{\textbf{\textit{#1}}}}
\newcommand{\OtherTok}[1]{\textcolor[rgb]{0.56,0.35,0.01}{#1}}
\newcommand{\FunctionTok}[1]{\textcolor[rgb]{0.00,0.00,0.00}{#1}}
\newcommand{\VariableTok}[1]{\textcolor[rgb]{0.00,0.00,0.00}{#1}}
\newcommand{\ControlFlowTok}[1]{\textcolor[rgb]{0.13,0.29,0.53}{\textbf{#1}}}
\newcommand{\OperatorTok}[1]{\textcolor[rgb]{0.81,0.36,0.00}{\textbf{#1}}}
\newcommand{\BuiltInTok}[1]{#1}
\newcommand{\ExtensionTok}[1]{#1}
\newcommand{\PreprocessorTok}[1]{\textcolor[rgb]{0.56,0.35,0.01}{\textit{#1}}}
\newcommand{\AttributeTok}[1]{\textcolor[rgb]{0.77,0.63,0.00}{#1}}
\newcommand{\RegionMarkerTok}[1]{#1}
\newcommand{\InformationTok}[1]{\textcolor[rgb]{0.56,0.35,0.01}{\textbf{\textit{#1}}}}
\newcommand{\WarningTok}[1]{\textcolor[rgb]{0.56,0.35,0.01}{\textbf{\textit{#1}}}}
\newcommand{\AlertTok}[1]{\textcolor[rgb]{0.94,0.16,0.16}{#1}}
\newcommand{\ErrorTok}[1]{\textcolor[rgb]{0.64,0.00,0.00}{\textbf{#1}}}
\newcommand{\NormalTok}[1]{#1}
\usepackage{longtable,booktabs}
\usepackage{graphicx,grffile}
\makeatletter
\def\maxwidth{\ifdim\Gin@nat@width>\linewidth\linewidth\else\Gin@nat@width\fi}
\def\maxheight{\ifdim\Gin@nat@height>\textheight\textheight\else\Gin@nat@height\fi}
\makeatother
% Scale images if necessary, so that they will not overflow the page
% margins by default, and it is still possible to overwrite the defaults
% using explicit options in \includegraphics[width, height, ...]{}
\setkeys{Gin}{width=\maxwidth,height=\maxheight,keepaspectratio}
\IfFileExists{parskip.sty}{%
\usepackage{parskip}
}{% else
\setlength{\parindent}{0pt}
\setlength{\parskip}{6pt plus 2pt minus 1pt}
}
\setlength{\emergencystretch}{3em}  % prevent overfull lines
\providecommand{\tightlist}{%
  \setlength{\itemsep}{0pt}\setlength{\parskip}{0pt}}
\setcounter{secnumdepth}{5}
% Redefines (sub)paragraphs to behave more like sections
\ifx\paragraph\undefined\else
\let\oldparagraph\paragraph
\renewcommand{\paragraph}[1]{\oldparagraph{#1}\mbox{}}
\fi
\ifx\subparagraph\undefined\else
\let\oldsubparagraph\subparagraph
\renewcommand{\subparagraph}[1]{\oldsubparagraph{#1}\mbox{}}
\fi

%%% Use protect on footnotes to avoid problems with footnotes in titles
\let\rmarkdownfootnote\footnote%
\def\footnote{\protect\rmarkdownfootnote}

%%% Change title format to be more compact
\usepackage{titling}

% Create subtitle command for use in maketitle
\newcommand{\subtitle}[1]{
  \posttitle{
    \begin{center}\large#1\end{center}
    }
}

\setlength{\droptitle}{-2em}
  \title{THE ULTIMATE GUIDE TO DATA MANIPULATION WITH R AND PYTHON}
  \pretitle{\vspace{\droptitle}\centering\huge}
  \posttitle{\par}
  \author{Zakaria Al Azhar, bigdatahabits.com}
  \preauthor{\centering\large\emph}
  \postauthor{\par}
  \predate{\centering\large\emph}
  \postdate{\par}
  \date{2017-11-12}

\usepackage{booktabs}

\usepackage{amsthm}
\newtheorem{theorem}{Theorem}[chapter]
\newtheorem{lemma}{Lemma}[chapter]
\theoremstyle{definition}
\newtheorem{definition}{Definition}[chapter]
\newtheorem{corollary}{Corollary}[chapter]
\newtheorem{proposition}{Proposition}[chapter]
\theoremstyle{definition}
\newtheorem{example}{Example}[chapter]
\theoremstyle{definition}
\newtheorem{exercise}{Exercise}[chapter]
\theoremstyle{remark}
\newtheorem*{remark}{Remark}
\newtheorem*{solution}{Solution}
\begin{document}
\maketitle

{
\setcounter{tocdepth}{1}
\tableofcontents
}
\chapter{Klm}\label{klm}

\chapter{Introduction}\label{intro}

\chapter{Prerequisites}\label{prerequisites}

\section{Basic knowledge}\label{basic-knowledge}

\section{Dataset}\label{dataset}

\section{Python and R packages}\label{python-and-r-packages}

\chapter{Data Exploration}\label{data-exploration}

\section{Data Structure}\label{data-structure}

\section{Data Summary}\label{data-summary}

\chapter{Column Formulas}\label{column-formulas}

After obtaining a good overview of the data, we can move to the next
step: manipulating data. In this chapter we present the most used data
manipulaton formulas on one ore more columns.

\section{Data Binning}\label{data-binning}

Data Binning is about grouping data in intervals - called bins. For
example, in the titanic dataset we've measured the age in years, but you
wanted to have age categories as follows:

\begin{itemize}
\tightlist
\item
  1 = Child , age ranges of 0-17
\item
  2 = Adult, age ranges of 18-39
\item
  3 = Middle Aged, age ranges of 40-59
\item
  4 = Over 60, age ranges of 60 and above
\end{itemize}

\textbf{R}

\begin{Shaded}
\begin{Highlighting}[]
\NormalTok{titanic =}\StringTok{ }\KeywordTok{read.csv}\NormalTok{(}\StringTok{"titanic.csv"}\NormalTok{)}
\CommentTok{#define the left  edges of the age categories and the corresponding labels:}
\NormalTok{edges <-}\StringTok{ }\KeywordTok{c}\NormalTok{(}\DecValTok{0}\NormalTok{,}\DecValTok{18}\NormalTok{,}\DecValTok{40}\NormalTok{,}\DecValTok{60}\NormalTok{, }\DecValTok{120}\NormalTok{)}
\NormalTok{labels <-}\StringTok{ }\KeywordTok{c}\NormalTok{(}\StringTok{"Child"}\NormalTok{,}\StringTok{"Adult"}\NormalTok{,}\StringTok{"Middle Aged"}\NormalTok{,}\StringTok{"Over 60"}\NormalTok{)}
\CommentTok{# we can break the ages in categories with the cut function}
\NormalTok{age.categories <-}\StringTok{ }\KeywordTok{cut}\NormalTok{(titanic}\OperatorTok{$}\NormalTok{Age,}\DataTypeTok{breaks =}\NormalTok{ edges, }\DataTypeTok{right =} \OtherTok{FALSE}\NormalTok{, }\DataTypeTok{labels =}\NormalTok{ labels)}
\CommentTok{# print the first 50 age items and the corresponding age categories)}
\NormalTok{age.categories[}\DecValTok{1}\OperatorTok{:}\DecValTok{50}\NormalTok{]}
\end{Highlighting}
\end{Shaded}

\begin{verbatim}
##  [1] Adult       Adult       Adult       Adult       Adult      
##  [6] <NA>        Middle Aged Child       Adult       Child      
## [11] Child       Middle Aged Adult       Adult       Child      
## [16] Middle Aged Child       <NA>        Adult       <NA>       
## [21] Adult       Adult       Child       Adult       Child      
## [26] Adult       <NA>        Adult       <NA>        <NA>       
## [31] Middle Aged <NA>        <NA>        Over 60     Adult      
## [36] Middle Aged <NA>        Adult       Adult       Child      
## [41] Middle Aged Adult       <NA>        Child       Adult      
## [46] <NA>        <NA>        <NA>        <NA>        Adult      
## Levels: Child Adult Middle Aged Over 60
\end{verbatim}

\textbf{PYTHON}

\begin{Shaded}
\begin{Highlighting}[]
\ImportTok{import}\NormalTok{ pandas }\ImportTok{as}\NormalTok{ pd}
\NormalTok{titanic }\OperatorTok{=}\NormalTok{ pd.read_csv(}\StringTok{"titanic.csv"}\NormalTok{)}
\NormalTok{labels }\OperatorTok{=}\NormalTok{ [}\StringTok{"Child"}\NormalTok{,}\StringTok{"Adult"}\NormalTok{,}\StringTok{"Middle Aged"}\NormalTok{, }\StringTok{"Over 60"}\NormalTok{]}
\NormalTok{edges }\OperatorTok{=}\NormalTok{ [}\DecValTok{0}\NormalTok{,}\DecValTok{18}\NormalTok{,}\DecValTok{40}\NormalTok{,}\DecValTok{60}\NormalTok{, }\DecValTok{120}\NormalTok{]}
\NormalTok{age_categories }\OperatorTok{=}\NormalTok{  pd.cut(titanic[}\StringTok{"Age"}\NormalTok{], edges, labels}\OperatorTok{=}\NormalTok{labels)}
\BuiltInTok{print}\NormalTok{ age_categories[}\DecValTok{0}\NormalTok{:}\DecValTok{20}\NormalTok{]}
\end{Highlighting}
\end{Shaded}

\begin{verbatim}
## 0           Adult
## 1           Adult
## 2           Adult
## 3           Adult
## 4           Adult
## 5             NaN
## 6     Middle Aged
## 7           Child
## 8           Adult
## 9           Child
## 10          Child
## 11    Middle Aged
## 12          Adult
## 13          Adult
## 14          Child
## 15    Middle Aged
## 16          Child
## 17            NaN
## 18          Adult
## 19            NaN
## Name: Age, dtype: category
## Categories (4, object): [Child < Adult < Middle Aged < Over 60]
\end{verbatim}

\section{Convert \& Replace}\label{convert-replace}

Convert \& Replace is a set of formulas that deal with converting and
replacing columns or individual cells.

\subsection{Category to Number}\label{category-to-number}

Category To Number is about converting nominal data to integer. Very
often, prediction or machine learning functions don't accept nominal
data, making it necessary to convert the field to integer if you want to
make predictions. For instance, the column `Sex' in the titanic dataset
is nominal consisting of ``male'' and ``female'', which can be encoded
to the integers 0/1, as follows:

\textbf{R}

\begin{Shaded}
\begin{Highlighting}[]
\NormalTok{titanic =}\StringTok{ }\KeywordTok{read.csv}\NormalTok{(}\StringTok{"titanic.csv"}\NormalTok{)}
\CommentTok{#define the left  edges of the age categories and the corresponding labels:}
\NormalTok{gender.encoded <-}\StringTok{ }\KeywordTok{as.integer}\NormalTok{(}\KeywordTok{as.factor}\NormalTok{(titanic}\OperatorTok{$}\NormalTok{Sex))}\OperatorTok{-}\DecValTok{1}
\CommentTok{#print subset}
\KeywordTok{head}\NormalTok{(gender.encoded)}
\end{Highlighting}
\end{Shaded}

\begin{verbatim}
## [1] 1 0 0 0 1 1
\end{verbatim}

\textbf{Python}

\begin{Shaded}
\begin{Highlighting}[]
\ImportTok{import}\NormalTok{ pandas }\ImportTok{as}\NormalTok{ pd}
\NormalTok{titanic }\OperatorTok{=}\NormalTok{ pd.read_csv(}\StringTok{"titanic.csv"}\NormalTok{)}
\NormalTok{gender_encoded }\OperatorTok{=}\NormalTok{   pd.Categorical(titanic.Sex).codes }
\BuiltInTok{print}\NormalTok{ gender_encoded[}\DecValTok{0}\NormalTok{:}\DecValTok{6}\NormalTok{]}
\end{Highlighting}
\end{Shaded}

\begin{verbatim}
## [1 0 0 0 1 1]
\end{verbatim}

\chapter{Final Words and this is
strange}\label{final-words-and-this-is-strange}

\chapter{Placeholder}\label{placeholder}

\bibliography{packages.bib,book.bib}


\end{document}
